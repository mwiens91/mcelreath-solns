% Set up the document
\documentclass{article}

% Page size
\usepackage[
    letterpaper,]{geometry}

% Lines between paragraphs
\setlength{\parskip}{\baselineskip}
\setlength{\parindent}{0pt}

% Math
\usepackage{amsmath}

% Links
\usepackage{hyperref}

\begin{document}

\section*{2. Display costs}

When considering the Hawk-Dove game in Chapter 2, Doves were able to
``display'' at no cost, and hence their payoff matrix was given by

\begin{center}
\begin{tabular}{ccc}
    & Hawk & Dove \\
    \hline
    Hawk & $(v - c) / 2$ & $v$ \\
    Dove & $0$ & $v / 2$ \\
    \hline
\end{tabular}
\end{center}

Now let's add a cost to displaying, $d / 2$, where $0 \leq d \leq v$.
Hence the modified payoff matrix is given by

\begin{center}
\begin{tabular}{ccc}
    & Hawk & Dove \\
    \hline
    Hawk & $(v - c) / 2$ & $v$ \\
    Dove & $0$ & $(v - d) / 2$ \\
    \hline
\end{tabular}
\end{center}

Note that when $d = 0$ the modified payoff matrix reduces to the
original payoff matrix; and that when $d = v$, two Doves in an
interaction over a resource incur no benefit from the interaction.

Following the methods in the textbook, let's show that neither Dove nor
Hawk is an ESS by considering what happens when a rare Hawk invades a
Dove population and what happens when a rare Dove invades a Hawk
population.

\textbf{Hawk invasion}

In a population of Doves, the fitness of a Dove is given by

\begin{equation*}
    W(D) \approx w_0 + \frac{v - d}{2}
\end{equation*}

and the fitness of a rare Hawk is given by

\begin{equation*}
    W(H) \approx w_0 + v
\end{equation*}

Hawks can invade a population of Doves if

\begin{align*}
    &W(H) > W(D) \\
    &\Rightarrow w_0 + v > w_0 + \frac{v - d}{2} \\
    &\Rightarrow v > - d
\end{align*}

which is always satisfied since $v > 0$ and $d > 0$. Hence Dove is not
an ESS.

\textbf{Dove invasion}

In a population of Hawks, the fitness of a Hawk is given by

\begin{equation*}
    W(H) \approx w_0 + \frac{v - c}{2}
\end{equation*}

and the fitness of a rare Dove is given by

\begin{equation*}
    W(D) \approx w_0
\end{equation*}

Doves can invade a population of Hawks if

\begin{align*}
    &W(D) > W(H) \\
    &\Rightarrow w_0 > w_0 + \frac{v - c}{2} \\
    &\Rightarrow c > v
\end{align*}

Thus when the cost of fighting over a resource is greater than the value
given by that resource, Doves can invade Hawks and Hawk is not an ESS.

\textbf{Mixed equilibrium}

Suppose there is a mixed equilibrium population of Doves and Hawks. At
this equilibrium, we must have $W(D) = W(H)$.

Now, if we let $p$ be the frequency of Hawks in the population, then
from the modified payoff matrix above we can read off that

\begin{align*}
    W(D) &= w_0 + (1 - p) \frac{v - d}{2} \\
    W(H) &= w_0 + p \frac{v - c}{2} + (1 - p) v
\end{align*}

Hence, letting $p = \hat{p}$ be the frequency of Hawks at the mixed
equilibrium, we have

\begin{align*}
    &W(D) = W(H) \\
    &\Rightarrow w_0 + (1 - \hat{p}) \frac{v - d}{2} = w_0 + \hat{p} \frac{v - c}{2} + (1 - \hat{p}) v \\
    &\Rightarrow (1 - \hat{p}) \frac{v - d}{2} = \hat{p} \frac{v - c}{2} + (1 - \hat{p}) v \\
    &\Rightarrow (\hat{p} - 1) \frac{v + d}{2} = \hat{p} \frac{v - c}{2} \\
    &\Rightarrow \hat{p} \frac{c + d}{2} = \frac{v + d}{2} \\
    &\Rightarrow \hat{p} = \frac{v + d}{c + d} \\
    &\Rightarrow \hat{p} = \frac{v}{c} + \frac{d (c - v)}{c (c + d)}
\end{align*}

With the final expression for $\hat{p}$, we see that when $d = 0$, the
$\hat{p}$ reduces to $v/c$, which is the value we obtain from the
original payoff matrix. When $d = v$, we can see that the frequency of
Doves is still non-zero. Hence a mixed equilibrium exists regardless of
the value of $d$ (given that it satisfied its original constraints),
which should be expected since we already concluded that Doves can
invade Hawks (provided $c > v$).

\end{document}
