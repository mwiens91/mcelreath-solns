% Set up the document
\documentclass{article}

% Page size
\usepackage[
    letterpaper,]{geometry}

% Lines between paragraphs
\setlength{\parskip}{\baselineskip}
\setlength{\parindent}{0pt}

% Math
\usepackage{amsmath}

% Links
\usepackage{hyperref}

\begin{document}

\section*{3. Social learning and replicator dynamics}

Consider a population practicing two alternate behaviors A and B.
Without loss of generality, assume behaviour A has a higher payoff than
behaviour B. We're told when two individuals practicing different
behaviors interact, each individual will adopt behaviour A with
probability $1 - e$ and behaviour B with probability $e$; and when two
individuals practicing the same behaviour meet, their behaviour does not
change. Hence the probabilities that an individual in an interaction
will adopt behaviour A \textit{after} the interaction are given by

\begin{align*}
    P(A|A,A) &= 1 \\
    P(A|A,B) &= 1 - e \\
    P(A|B,A) &= 1 - e \\
    P(A|B,B) &= 0
\end{align*}

Letting $p$ be the probability that an individual in a population
practices behaviour A, we have that the probabilities of each type of
interaction are given by

\begin{align*}
    P(A,A) &= p^2 \\
    P(A,B) &= p(1 - p) \\
    P(B,A) &= p(1 - p) \\
    P(B,B) &= (1 - p)^2
\end{align*}

Hence the value of $p$ after the first phase of the population's
interactions, $p'$, is given by

\begin{align*}
    p' &= P(A,A)P(A|A,A) + P(A,B)P(A|A,B) + P(B,A)P(A|B,A) + P(B,B)P(A|B,B) \\
       &= p^2 \cdot 1 + 2p(1 - p) \cdot (1 - e) + (1 - p)^2 \cdot 0 \\
       &= p^2 + 2p(1 - p)(1 - e) \\
       &= (2e - 1)p^2 + 2(1 - e)p
\end{align*}

and the change in frequency of behaviour A, $\Delta p$, is given by

\begin{align*}
    \Delta p &= p' - p \\
             &= \{(2e - 1)p^2 + 2(1 - e)p\} - p \\
             &= (2e - 1)p^2 + (1 - 2e)p \\
             &= p(1 - p)(1 - 2e)
\end{align*}

\section*{6. Horizontal cultural transmission, again}

Consider a population practicing two alternate behaviours A and B. The
payoff of behaviours A and B are $V(A)$ and $V(B)$, respectively. When
two individuals interact, with one practicing behaviour A and the other
practicing behaviour B, the individual practicing behaviour B will
switch to behaviour A after the interaction with a probability
proportional to $\beta (V(A) - V(B))$; otherwise, individuals in an
interaction will not switch behaviours.

The probabilities that an individual in an interaction will adopt
behaviour A after the interaction are given by

\begin{align*}
    P(A|A,A) &= 1 \\
    P(A|A,B) &= 1 \\
    P(A|B,A) &= c \beta (V(A) - V(B)) \\
    P(A|B,B) &= 0
\end{align*}

where $c$ is a constant of proportionality.

Letting $p$ be the probability that an individual in a population
practices behaviour A, we again have that the probabilities of each type
of interaction are given by

\begin{align*}
    P(A,A) &= p^2 \\
    P(A,B) &= p(1 - p) \\
    P(B,A) &= p(1 - p) \\
    P(B,B) &= (1 - p)^2
\end{align*}

Hence the value of $p$ after the first phase of the population's
interactions, $p'$, is given by

\begin{align*}
    p' &= P(A,A)P(A|A,A) + P(A,B)P(A|A,B) + P(B,A)P(A|B,A) + P(B,B)P(A|B,B) \\
       &= p^2 \cdot 1
          + p(1 - p) \cdot 1
          + p(1 - p) \cdot c \beta (V(A) - V(B))
          + (1 - p)^2 \cdot 0 \\
       &= p + c \beta p(1 - p)(V(A) - V(B))
\end{align*}

The change in frequency of behaviour A, $\Delta p$, is given by

\begin{align*}
    \Delta p &= p' - p \\
             &= \{p + c \beta p(1 - p)(V(A) - V(B))\} - p \\
             &= c \beta p(1 - p)(V(A) - V(B))
\end{align*}

If $c = \frac{1}{p(1 - p)}$ (why should this be the case?) then

\begin{equation*}
    \Delta p = \beta (V(A) - V(B))
\end{equation*}

\end{document}
