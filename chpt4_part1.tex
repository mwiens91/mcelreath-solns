% Set up the document
\documentclass{article}

% Page size
\usepackage[
    letterpaper,]{geometry}

% Lines between paragraphs
\setlength{\parskip}{\baselineskip}
\setlength{\parindent}{0pt}

% Math
\usepackage{amsfonts}
\usepackage{amsmath}

% Links
\usepackage{hyperref}

\begin{document}

\section*{2. Even bad guys are vulnerable}

Here we are considering a population where individuals interact with
each other in an iterated prisoner's dilemma. Recall that the strategy
ALLC (always cooperate) will always cooperate; ALLD (always defect) will
always defect; TFT (tit-for-tat) will always cooperate in the first
round and will cooperate in subsequent rounds provided their opponent
did not defect in the preceding round (otherwise they will defect); STFT
(suspicious tit-for-tat) will defect in the first round and play the
same as TFT for subsequent rounds; and TF2T (tit-for-two-tats) will
cooperate in the first two rounds and will cooperate in subsequent
rounds provided their opponent did not defect in both of the preceding
two rounds (otherwise they will defect).

We want to show that a population in which ALLD is common can be invaded
by STFT when mutation maintains TF2T in the population at a low
frequency. Let $p$ and $q$ be the frequency of ALLD and TF2T in the
population, respectively. Letting $W(i)$ be the fitness of strategy $i$
in this population, we need to show that

\begin{align*}
    W(STFT) &> W(ALLD) \\
    W(STFT) &> W(TF2T)
\end{align*}

Both of these inequalities provide a sufficient condition (though not a
necessary condition) for STFT to invade. Let's go ahead and find the
fitnesses of each strategy. To do so, we need to first determine the
payoffs $V(i|j)$ for strategy $i$ against strategy $j$:

For STFT we have

\begin{align*}
    V(STFT|ALLD) &= 0 \\
    V(STFT|TF2T) &= b + w V(ALLC|ALLC) \\
                 &= b + w \frac{b - c}{1 - w} \\
    V(STFT|STFT) &= 0
\end{align*}

For ALLD we have

\begin{align*}
    V(ALLD|ALLD) &= 0 \\
    V(ALLD|TF2T) &= 2 b \\
    V(ALLD|STFT) &= 0
\end{align*}

And for TF2T we have

\begin{align*}
    V(TF2T|ALLD) &= - 2 c \\
    V(TF2T|TF2T) &= V(ALLC|ALLC) \\
                 &= \frac{b - c}{1 - w} \\
    V(TF2T|STFT) &= - c + w V(ALLC|ALLC) \\
                 &= - c + w \frac{b - c}{1 - w}
\end{align*}

Hence the fitnesses $W(i)$ are

\begin{align*}
    W(STFT) &= w_0 + p V(STFT|ALLD) + q V(STFT|TF2T) + (1 - p - q) V(STFT|STFT) \\
            &= w_0 + q \Big(b + w \frac{b - c}{1 - w}\Big) \\
    W(ALLD) &= w_0 + p V(ALLD|ALLD) + q V(ALLD|TF2T) + (1 - p - q) V(ALLD|STFT) \\
            &= w_0 + 2 b q \\
    W(TF2T) &= w_0 + p V(TF2T|ALLD) + q V(TF2T|TF2T) + (1 - p - q) V(TF2T|STFT) \\
            &= w_0 + - 2 c p + q \frac{b - c}{1 - w} + (1 - p - q) \Big(- c + w \frac{b - c}{1 - w}\Big)
\end{align*}

Now let's go back to the sufficient condition that we're interested in.
We start with pitting the fitness of STFT against the fitness of ALLD:

\begin{align*}
    &W(STFT) > W(ALLD) \\
    &\Rightarrow w_0 + q \Big(b + w \frac{b - c}{1 - w}\Big) > w_0 + 2 b q \\
    &\Rightarrow b + w \frac{b - c}{1 - w} > 2 b \\
    &\Rightarrow w \frac{b - c}{1 - w} > b \\
\end{align*}

The last inequality tells us that if the benefit of cooperating in
subsequent rounds (after defecting the first round) with a TF2T is
greater than the benefit of defecting the first two rounds with a TF2T
(and never cooperating thereafter), then STFT will have higher fitness
than ALLD. It all comes down to how STFT and ALLD play against TF2T: if
we leave the payoffs in the form $V(i|j)$ then the above inequalities
are equivalent to

\begin{equation*}
    V(STFT|TF2T) > V(ALLD|TF2T)
\end{equation*}

Now comparing the fitness of STFT against TF2T, we have

\begin{align*}
    &W(STFT) > W(TF2T) \\
    &\Rightarrow w_0 + q \Big(b + w \frac{b - c}{1 - w}\Big) > w_0 + - 2 c p + q \frac{b - c}{1 - w} + (1 - p - q) \Big(- c + w \frac{b - c}{1 - w}\Big) \\
    &\Rightarrow q \Big(b + w \frac{b - c}{1 - w}\Big) > - 2 c p + q \frac{b - c}{1 - w} + (1 - p - q) \Big(- c + w \frac{b - c}{1 - w}\Big)
\end{align*}

If we make the $p \approx 1$ (and hence $q \approx 0$) (note this is
appropriate because $p \gg q$ by assumption), then the above inequality
reduces to

\begin{equation*}
    0 > - 2 c
\end{equation*}

which is clearly true. Without approximating, the exact inequalities
simplify to

\begin{equation*}
    q < \frac{(1 + p - 2 p w) c - (1 - p) b}{w (b - c)}
\end{equation*}

While that this exact inequality isn't particularly informative, it does
tell us that at values of $q$ greater than or equal to the RHS, our
sufficient condition necessarily breaks: so we require $q$ to be small
but still non-zero---else $W(STFT) = W(ALLD)$ and STFT cannot invade.

Hence STFT can always invade in this scenario provided that they play
better against TF2T than ALLD does \textit{and} that the frequency of
TF2T is sufficiently small but still non-zero.

\section*{4. Doing the dishes again, and again}

Consider an iterated game involving two players either resting (R) or
washing dishes (W) with payoffs

\begin{align*}
    V(R|R) &= 0 \\
    V(R|W) &= b \\
    V(W|W) &= b - \frac{k}{2} \\
    V(W|R) &= b - k
\end{align*}

where $b, k > 0$. After each round, the players continue playing the
game with probability $w$.

Consider three strategies: AW (always wash) always washes; NW (never
wash) never washes; and WFT (wash-for-tat), which has the following
behaviour: on the first round if the opponent insists on washing, WFT
will rest; if the opponent insists on resting, WFT will wash; and if the
opponent is another WFT, one WFT will wash or rest at random, and the
other WFT will play the action not chosen by the first. On subsequent
rounds, WFT will play the action that their opponent played in the
preceding round.

The first result we want to show is

\begin{equation*}
    V(WFT|WFT) = V(AW|AW) = \frac{b - k / 2}{1 - w}
\end{equation*}

Let's calculate each of the payoffs:

\begin{align*}
    V(AW|AW) &= \sum_{i = 0}^{\infty} V(W|W) w^i \\
             &= V(W|W) \frac{1}{1 - w} \\
             &= \frac{b - k / 2}{1 - w}
\end{align*}

Using the notation where $2 \mathbb{Z}$ denotes the set of even integers
and $2 \mathbb{Z} + 1$ denotes the set of odd integers, we have

\begin{align*}
    V(WFT|WFT) &=
        \frac{1}{2} \Bigg[
            \sum_{\substack{i = 0 \\ i \in 2 \mathbb{Z}}}^{\infty} V(R|W) w^i
            + \sum_{\substack{i = 0 \\ i \in 2 \mathbb{Z} + 1}}^{\infty} V(W|R) w^i
        \Bigg]
        + \frac{1}{2} \Bigg[
            \sum_{\substack{i = 0 \\ i \in 2 \mathbb{Z}}}^{\infty} V(W|R) w^i
            + \sum_{\substack{i = 0 \\ i \in 2 \mathbb{Z} + 1}}^{\infty} V(R|W) w^i
        \Bigg] \\
               &= \frac{1}{2} \sum_{i = 0}^{\infty} V(R|W) w^i + \frac{1}{2} \sum_{i = 0}^{\infty} V(W|R) w^i \\
               &= \frac{1}{2} \Big(V(R|W) + V(W|R)\Big) \frac{1}{1 - w} \\
               &= \frac{1}{2} (b + b - k) \frac{1}{1 - w} \\
               &= \frac{b - k / 2}{1 - w}
\end{align*}

Hence we indeed have that

\begin{equation*}
    V(WFT|WFT) = V(AW|AW) = \frac{b - k / 2}{1 - w}
\end{equation*}

The next (and final) result we want to show is that WFT is stable
against invasion by rare NW when

\begin{equation*}
    w b > k / 2
\end{equation*}

First let's calculate $V(NW|WFT)$:

\begin{align*}
    V(NW|WFT) &= V(R|W) + w \sum_{i = 0}^{\infty} V(R|R) w^i \\
              &= b
\end{align*}

Hence the fitnesses of both strategies in a population of WFTs are given by

\begin{align*}
    W(WFT) &= w_0 + \frac{b - k / 2}{1 - w} \\
    W(NW) &= w_0 + b
\end{align*}

WFT will be stable against invasion by rare NW provided

\begin{align*}
    &W(WFT) > W(NW) \\
    &\Rightarrow w_0 + \frac{b - k / 2}{1 - w} > w_0 + b \\
    &\Rightarrow b - k / 2 > (1 - w) b \\
    &\Rightarrow w b > k / 2
\end{align*}

which is the result we wanted to show.

Now, Hamilton's rule states that (in the context of the prisoner's dilemma)

\begin{equation*}
    b r > c
\end{equation*}

where $b$ is the benefit for having your opponent cooperate and $c$ is
the cost of cooperating. This statement is very similar to the last
result we derived. The $b$ parameters are shared and have the same
meaning, and we can identify $c$ with $k / 2$, again with the same
meaning. $w$ and $r$ have different meanings, but play the same role in
terms of the equations: $r$ tells you how correlated your altruistic
behaviour is with others' altruistic behaviour, and $w$ tells you how
likely altruists are to continue interacting with each other---both tell
you the extent to which altruism begets altruism.

\end{document}
