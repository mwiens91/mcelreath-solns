% Set up the document
\documentclass{article}

% Page size
\usepackage[
    letterpaper,]{geometry}

% Lines between paragraphs
\setlength{\parskip}{\baselineskip}
\setlength{\parindent}{0pt}

% Math
\usepackage{amsmath}

% Links
\usepackage{hyperref}

\begin{document}

\section*{2. Covariance assortment}

In Chapter 3 we saw a general non-genetic form of Hamilton's rule
expressed in terms of conditional probabilities of interaction:

\begin{equation*}
    \{P(C|C) - Pr(C|D)\} b > c
\end{equation*}

where $P(C|C)$ is the probability a ``cooperator'' meets a cooperator,
$P(C|D)$ is the probability a ``defector'' meets a cooperator, and $b$
and $c$ are constants representing the value of receiving aid and the
cost of giving aid, respectively.

We need to show that the most general covariance form of Hamilton's rule,

\begin{equation*}
    \frac{cov(p_i, y_i)}{cov(p_i, h_i)} b > c
\end{equation*}

reduces to the top-most expression for Hamilton's rule when assuming a
haploid model where individuals with altruism alleles always act
altruistically and individuals without altruism alleles never act
altruistically.

Note that in the above expression $p_i$, $y_i$, and $h_i$ represent, for
the $i$th individual, the frequency of the altruism allele in the
individual (this is binary-valued by assumption), the probability that
they receive aid, and the probability that they give aid, respectively.

Expanding out the covariances above we have

\begin{align*}
    cov(p_i, y_i) &= E(p_i y_i) - E(p_i) E(y_i) \\
    cov(p_i, h_i) &= E(p_i h_i) - E(p_i) E(h_i)
\end{align*}

Let $p$ be the frequency of the altruism allele in the population.

Note that $E(p_i) = p$: the expectation of the frequency if the altruism
allele over all $i$th individuals is just the frequency of the altruism
allele in the population. Also note that $h_i = p_i$: the probability
the $i$th individual gives aid is just the frequency of the altruism
allele in that individual. These ``notes'' jointly imply $E(h_i) = p$.

Hence we can simplify our above covariance expressions as follows:

\begin{align*}
    cov(p_i, y_i) &= E(p_i y_i) - p E(y_i) \\
    cov(p_i, h_i) &= E(p_i^2) - p^2
\end{align*}

Noting that $E(p_i^2) = p \cdot 1^2 + (1 - p) \cdot 0^2 = p$, we can
simplify even further:

\begin{align*}
    cov(p_i, y_i) &= E(p_i y_i) - p E(y_i) \\
    cov(p_i, h_i) &= p (1 - p)
\end{align*}

Now, we can express $y_i$ in terms of $p_i$ by breaking up the cases in
which the $i$th individual is an altruist or a defector:

\begin{equation*}
    y_i = p_i P(C|C) + (1 - p_i) P(C|D)
\end{equation*}

This implies, using $E(p_i)$ above,

\begin{equation*}
    E(y_i) = p P(C|C) + (1 - p) P(C|D)
\end{equation*}

and also

\begin{align*}
    E(p_i y_i) &= p \cdot 1 \cdot 1 \cdot P(C|C) + (1 - p) \cdot 0 \cdot 1 \cdot P(C|D) \\
               &= p P(C|C)
\end{align*}

Hence, going back to the expressions for our covariances, we have

\begin{align*}
    cov(p_i, y_i) &= p \{P(C|C) - [p P(C|C) + (1 - p) P(C|D)]\} \\
    cov(p_i, h_i) &= p (1 - p)
\end{align*}

Therefore, we have that

\begin{align*}
    &\frac{cov(p_i, y_i)}{cov(p_i, h_i)} b > c \\
    &\Rightarrow \frac{p \{P(C|C) - [p P(C|C) + (1 - p) P(C|D)]\}}{p (1 - p)} b > c \\
    &\Rightarrow \frac{P(C|C) - [p P(C|C) + (1 - p) P(C|D)]}{(1 - p)} b > c \\
    &\Rightarrow \{P(C|C) - P(C|D)\} b > c
\end{align*}

which is desired form of Hamilton's rule.

\end{document}
