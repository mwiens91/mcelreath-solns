% Set up the document
\documentclass{article}

% Page size
\usepackage[
    letterpaper,]{geometry}

% Lines between paragraphs
\setlength{\parskip}{\baselineskip}
\setlength{\parindent}{0pt}

% Math
\usepackage{amsmath}

% Links
\usepackage{hyperref}

\begin{document}

\section*{2. Covariance assortment}

In Chapter 3 we saw a general non-genetic form of Hamilton's rule
expressed in terms of conditional probabilities of interaction:

\begin{equation*}
    \{P(C|C) - Pr(C|D)\} b > c
\end{equation*}

where $P(C|C)$ is the probability a ``cooperator'' meets a cooperator,
$P(C|D)$ is the probability a ``defector'' meets a cooperator, and $b$
and $c$ are constants representing the value of receiving aid and the
cost of giving aid, respectively.

We need to show that the most general covariance form of Hamilton's rule,

\begin{equation*}
    \frac{cov(p_i, y_i)}{cov(p_i, h_i)} b > c
\end{equation*}

reduces to the top-most expression for Hamilton's rule when assuming a
haploid model where individuals with altruism alleles always act
altruistically and individuals without altruism alleles never act
altruistically.

Note that in the above expression $p_i$, $y_i$, and $h_i$ represent, for
the $i$th individual, the frequency of the altruism allele in the
individual (this is binary-valued by assumption), the probability that
they receive aid, and the probability that they give aid, respectively.

Expanding out the covariances above we have

\begin{align*}
    cov(p_i, y_i) &= E(p_i y_i) - E(p_i) E(y_i) \\
    cov(p_i, h_i) &= E(p_i h_i) - E(p_i) E(h_i)
\end{align*}

Let $p$ be the frequency of the altruism allele in the population.

Note that $E(p_i) = p$: the expectation of the frequency if the altruism
allele over all $i$th individuals is just the frequency of the altruism
allele in the population. Also note that $h_i = p_i$: the probability
the $i$th individual gives aid is just the frequency of the altruism
allele in that individual. These ``notes'' jointly imply $E(h_i) = p$.

Hence we can simplify our above covariance expressions as follows:

\begin{align*}
    cov(p_i, y_i) &= E(p_i y_i) - p E(y_i) \\
    cov(p_i, h_i) &= E(p_i^2) - p^2
\end{align*}

Noting that $E(p_i^2) = p \cdot 1^2 + (1 - p) \cdot 0^2 = p$, we can
simplify even further:

\begin{align*}
    cov(p_i, y_i) &= E(p_i y_i) - p E(y_i) \\
    cov(p_i, h_i) &= p (1 - p)
\end{align*}

Now, we can express $y_i$ in terms of $p_i$ by breaking up the cases in
which the $i$th individual is an altruist or a defector:

\begin{equation*}
    y_i = p_i P(C|C) + (1 - p_i) P(C|D)
\end{equation*}

This implies, using $E(p_i)$ above,

\begin{equation*}
    E(y_i) = p P(C|C) + (1 - p) P(C|D)
\end{equation*}

and also

\begin{align*}
    E(p_i y_i) &= p \cdot 1 \cdot 1 \cdot P(C|C) + (1 - p) \cdot 0 \cdot 1 \cdot P(C|D) \\
               &= p P(C|C)
\end{align*}

Hence, going back to the expressions for our covariances, we have

\begin{align*}
    cov(p_i, y_i) &= p \{P(C|C) - [p P(C|C) + (1 - p) P(C|D)]\} \\
    cov(p_i, h_i) &= p (1 - p)
\end{align*}

Therefore, we have that

\begin{align*}
    &\frac{cov(p_i, y_i)}{cov(p_i, h_i)} b > c \\
    &\Rightarrow \frac{p \{P(C|C) - [p P(C|C) + (1 - p) P(C|D)]\}}{p (1 - p)} b > c \\
    &\Rightarrow \frac{P(C|C) - [p P(C|C) + (1 - p) P(C|D)]}{(1 - p)} b > c \\
    &\Rightarrow \{P(C|C) - P(C|D)\} b > c
\end{align*}

which is desired form of Hamilton's rule.

\section*{4. Frequency dependent assortment}

Here we consider altruistic and selfish behaviour with the following
payoffs (A denotes altruistic behaviour and S denotes selfish
behaviour):

\begin{center}
\begin{tabular}{ccc}
    & A & S \\
    \hline
    A & $b - c$ & $-c$ \\
    S & $b$ & $0$ \\
    \hline
\end{tabular}
\end{center}

where the payoffs listed are for the ``row player''.

Assume $b - c > 0$. Consider limited dispersal where the probabilities
of each possible interaction are given by

\begin{align*}
    P(A,A) &= \frac{p^2 (1 + a)}{1 + a p} \\
    P(S,S) &= \frac{(1 - p^2) (1 + a)}{1 + a (1 - p)} \\
    P(A,S) &= p (1 - p) \bigg(\frac{1}{1 + a p} + \frac{1}{1 + a (1 - p)}\bigg)
\end{align*}

where $a$ is a positive constant and $p$ is the frequency of altruists
in the population. Note that higher values of $a$ mean more assortative
interaction.

The first thing we'll do is inject the above probabilities of
interactions into Hamilton's rule

\begin{equation*}
    \{P(A|A) - Pr(A|S)\} b > c
\end{equation*}

Now, using probability rules we have,

\begin{align*}
    P(A|A) &= \frac{P(A,A)}{P(A)} \\
           &= \frac{\frac{p^2 (1 + a)}{1 + a p}}{p} \\
           &= \frac{p (1 + a)}{1 + a p} \\ \\
    P(A|S) &= \frac{P(A,S)}{P(S)} \\
           &= \frac{p (1 - p) \Big(\frac{1}{1 + a p} + \frac{1}{1 + a (1 - p)}\Big)}{1 - p} \\
           &= p \bigg(\frac{1}{1 + a p} + \frac{1}{1 + a (1 - p)}\bigg)
\end{align*}

Thus, substituting $P(A|A)$ and $P(A|S)$ into Hamilton's rule above, we have

\begin{equation*}
    \Bigg\{\frac{p (1 + a)}{1 + a p} - p \bigg(\frac{1}{1 + a p} + \frac{1}{1 + a (1 - p)}\bigg) \Bigg\} b > c
\end{equation*}

which we can also express in the following form

\begin{equation*}
    \frac{p (1 + a^2 (p - 1) + a (p - 1))}{(a (p - 1) - 1)(a p + 1)} b > c
\end{equation*}

Now we're in a great position to answer a few biological questions:

(a) Is selfish behaviour evolutionarily stable? Yes! In this case we set
$p = 0$ and we can immediately see from Hamilton's rule that altruistic
behaviour it \textit{not} evolutionarily stable---which implies that
selfish behaviour \textit{is} evolutionarily stable.

(b) Is altruistic behaviour evolutionarily stable? No! Here we set $p =
1$, which, substituted into Hamilton's rule, gives

\begin{equation*}
    - \frac{1}{a + 1} b > c
\end{equation*}

which is clearly false, indicating that altruistic behaviour.

(c) Can altruistic behaviour occur at values of ``intermediate'' values
of $p$ such that $0 < p < 1$? Yes! So long as $a$ satisfies

\begin{equation*}
    a > \frac{
        -b p^2 + p b - c + \sqrt{
            b^2 p^4 - 6 b^2 p^3 + 4 b c p^3 + 5 b^2 p^2 -6 b c p^2 + 4 c^2 p^2 + 2 b c p - 4 c^2 p + c^2
        }}{2 p (p b - c p - b + c)}
\end{equation*}

While the above condition for $a$ is exact, it perhaps isn't so
illuminating. So let's develop a bit more intuition by taking the limit
of $a$ in the first term of our expression for Hamilton's rule:

\begin{equation*}
    \lim_{a\to\infty} \frac{p (1 + a^2 (p - 1) + a (p - 1))}{(a (p - 1) - 1)(a p + 1)} = 1
\end{equation*}

Hence

\begin{equation*}
    \lim_{a\to\infty} \{P(A|A) - Pr(A|S)\} b = b > c
\end{equation*}

where $b > c$ by assumption. Hence for $a$ sufficiently large we see
that Hamilton's rule is satisfied for intermediate values of $p$.

\end{document}
