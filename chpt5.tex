% Set up the document
\documentclass{article}

% Page size
\usepackage[
    letterpaper,]{geometry}
% Lines between paragraphs
\setlength{\parskip}{\baselineskip}
\setlength{\parindent}{0pt}

% Math
\usepackage{amsfonts}
\usepackage{amsmath}

% Links
\usepackage{hyperref}

\begin{document}

\section*{2. Sometimes it's better to say nothing at all}

Recall that in the discrete Sir Philip Sidney game the minimum-cost
signal which could allow stable, honest, communication had cost

\begin{equation*}
    \hat{c} = b - r d
\end{equation*}

\textbf{Part a}

For the first part of Part a, assume a population where Responder (R) is
common and needy Responders signal with cost $\hat{c}$. The first thing
we want to do is compute the fitness of donors in this population. Let
$p$ be the percentage of beneficiaries that are in severe need. Then, we
have

\begin{align*}
    W_D(R) &= w_0 + V_D(R|R) + r V_B(R|R) \\
           &= w_0
              + \left((1 - p) 1 + p (1 - d)\right)
              + r \left((1 - p) (1 - b) + p (1 - \hat{c})\right) \\
           &= w_0 + 1 - p d + r \left((1 - p) (1 - b) + p (1 - \hat{c})\right) \\
           &= w_0 + 1 - p d + r \left((1 - p) (1 - b) + p (1 - b + r d)\right) \\
           &= w_0 + 1 - p d + r \left(p d r - b + 1\right) \\
           &= w_0 + p d (r^2 - 1) + (1 - b) r + 1
\end{align*}

For the second part of Part a, assume a population where the strategy
Always Give and Never Signal (AGNS) is common. We want to show that the
donors in this non-signaling population have a higher fitness than
donors in the Responder population provided

\begin{equation*}
    b > \frac{d}{r} (1 - p) + p r d
\end{equation*}

To show this inequality, let's first determine the fitness of donors in
the AGNS population:

\begin{align*}
    W_D(AGNS) &= w_0 + V_D(AGNS|AGNS) + r V_B(AGNS|AGNS) \\
              &= w_0 + 1 - d + r
\end{align*}

Let's now derive the inequality. The donors in the non-signaling
population will have higher fitness provided

\begin{align*}
    &W_D(AGNS) > W_D(R) \\
    &\Rightarrow w_0 + 1 - d + r > w_0 + p d (r^2 - 1) + (1 - b) r + 1 \\
    &\Rightarrow - d > p d (r^2 - 1) - b r \\
    &\Rightarrow b > \frac{p d (r^2 - 1) + d}{r} \\
    &\Rightarrow b > \frac{d}{r} (1 - p) + p r d
\end{align*}

\textbf{Part b}

Now we want to repeat Part a, but consider the fitness of beneficiaries
of the two populations instead of donors. We want to show that
beneficiaries in the non-signaling population will have higher fitness
provided

\begin{equation*}
    b > r d
\end{equation*}

The fitnesses for the beneficiaries are given by

\begin{align*}
    W_B(R) &= w_0 + r V_D(R|R) + V_B(R|R) \\
           &= w_0 + r (1 - p d) + p d r - b + 1 \\
           &= w_0 + r - b + 1
\end{align*}

\begin{align*}
    W_B(AGNS) &= w_0 + r V_D(AGNS|AGNS) + V_B(AGNS|AGNS) \\
              &= w_0 + r (1 - d) + 1
\end{align*}

The beneficiaries in the non-signaling population will have higher
fitness provided

\begin{align*}
    &W_B(AGNS) > W_B(R) \\
    &\Rightarrow w_0 + r (1 - d) + 1 > w_0 + r - b + 1 \\
    &\Rightarrow - r d > - b \\
    &\Rightarrow b > r d
\end{align*}

\textbf{Part c}

Now we're asked to show that there exist choices of parameters such that
both signaling and non-signaling equilibria are stable and in which the
average fitness of both donors and beneficiaries is less in the
signaling population than in the non-signaling population.

\textbf{what are they even asking for here??}

\end{document}
