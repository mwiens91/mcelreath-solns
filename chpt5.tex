% Set up the document
\documentclass{article}

% Page size
\usepackage[
    letterpaper,]{geometry}
% Lines between paragraphs
\setlength{\parskip}{\baselineskip}
\setlength{\parindent}{0pt}

% Math
\usepackage{amsfonts}
\usepackage{amsmath}

% Links
\usepackage{hyperref}

% For decision tree
\usepackage[latin1]{inputenc}
\usepackage{tikz}
\usetikzlibrary{trees}

\begin{document}

\section*{2. Sometimes it's better to say nothing at all}

Recall that in the discrete Sir Philip Sidney game the minimum-cost
signal which could allow stable, honest, communication had cost

\begin{equation*}
    \hat{c} = b - r d
\end{equation*}

\textbf{Part a}

For the first part of Part a, assume a population where Responder (R) is
common and needy Responders signal with cost $\hat{c}$. The first thing
we want to do is compute the fitness of donors in this population. Let
$p$ be the percentage of beneficiaries that are in severe need. Then, we
have

\begin{align*}
    W_D(R) &= w_0 + V_D(R|R) + r V_B(R|R) \\
           &= w_0
              + \left((1 - p) 1 + p (1 - d)\right)
              + r \left((1 - p) (1 - b) + p (1 - \hat{c})\right) \\
           &= w_0 + 1 - p d + r \left((1 - p) (1 - b) + p (1 - \hat{c})\right) \\
           &= w_0 + 1 - p d + r \left((1 - p) (1 - b) + p (1 - b + r d)\right) \\
           &= w_0 + 1 - p d + r \left(p d r - b + 1\right) \\
           &= w_0 + p d (r^2 - 1) + (1 - b) r + 1
\end{align*}

For the second part of Part a, assume a population where the strategy
Always Give and Never Signal (AGNS) is common. We want to show that the
donors in this non-signaling population have a higher fitness than
donors in the Responder population provided

\begin{equation*}
    b > \frac{d}{r} (1 - p) + p r d
\end{equation*}

To show this inequality, let's first determine the fitness of donors in
the AGNS population:

\begin{align*}
    W_D(AGNS) &= w_0 + V_D(AGNS|AGNS) + r V_B(AGNS|AGNS) \\
              &= w_0 + 1 - d + r
\end{align*}

Let's now derive the inequality. The donors in the non-signaling
population will have higher fitness provided

\begin{align*}
    &W_D(AGNS) > W_D(R) \\
    &\Rightarrow w_0 + 1 - d + r > w_0 + p d (r^2 - 1) + (1 - b) r + 1 \\
    &\Rightarrow - d > p d (r^2 - 1) - b r \\
    &\Rightarrow b > \frac{p d (r^2 - 1) + d}{r} \\
    &\Rightarrow b > \frac{d}{r} (1 - p) + p r d
\end{align*}

\textbf{Part b}

Now we want to repeat Part a, but consider the fitness of beneficiaries
of the two populations instead of donors. We want to show that
beneficiaries in the non-signaling population will have higher fitness
provided

\begin{equation*}
    b > r d
\end{equation*}

The fitnesses for the beneficiaries are given by

\begin{align*}
    W_B(R) &= w_0 + r V_D(R|R) + V_B(R|R) \\
           &= w_0 + r (1 - p d) + p d r - b + 1 \\
           &= w_0 + r - b + 1
\end{align*}

\begin{align*}
    W_B(AGNS) &= w_0 + r V_D(AGNS|AGNS) + V_B(AGNS|AGNS) \\
              &= w_0 + r (1 - d) + 1
\end{align*}

The beneficiaries in the non-signaling population will have higher
fitness provided

\begin{align*}
    &W_B(AGNS) > W_B(R) \\
    &\Rightarrow w_0 + r (1 - d) + 1 > w_0 + r - b + 1 \\
    &\Rightarrow - r d > - b \\
    &\Rightarrow b > r d
\end{align*}

\textbf{Part c}

Now we're asked to show that there exist choices of parameters such that
both signaling and non-signaling equilibria are stable and in which the
average fitness of both donors and beneficiaries is less in the
signaling population than in the non-signaling population.

\textbf{what are they even asking for here??}

\section*{4. External commitment}

Consider the following kidnapping scenario: Joe has kidnapped Sue and
successfully received a randsom from her parents. Joe now needs to
decide whether to release or kill Sue. All other things equal, Joe
prefers not to kill Sue; but if Joe releases Sue, she may identify Joe
to the police.

Sue makes a promise to Joe that she will not identify Joe to the police
provided she is released. Should Joe believe her?

We determine this using backwards induction on a game tree, where the
respective payoffs to Joe and Sue are labeled on the leaf nodes:

% Set the overall layout of the tree
\tikzstyle{level 1}=[level distance=3.5cm, sibling distance=5cm]
\tikzstyle{level 2}=[level distance=3.5cm, sibling distance=5cm]
\tikzstyle{level 3}=[level distance=3.5cm, sibling distance=2.5cm]
\tikzstyle{level 4}=[level distance=3.5cm, sibling distance=2.5cm]

% Define styles for bags and leafs
\tikzstyle{bag} = [text width=9em, text centered]

% The sloped option gives rotated edge labels. Personally
% I find sloped labels a bit difficult to read. Remove the sloped options
% to get horizontal labels.
\begin{center}
\begin{tikzpicture}[grow=down, sloped]
\node[bag] {Joe chooses}
    child {
        node[bag] {safe, dead}
            edge from parent
            node[above] {Kill}
    }
    child {
        node[bag] {Sue chooses}
        child {
                node[bag] {unsafe, alive}
                edge from parent
                node[above] {Identify}
            }
        child {
                node[bag] {safe, alive}
                edge from parent
                node[above] {Don't identify}
            }
        edge from parent
            node[above] {Release}
    };
\end{tikzpicture}
\end{center}

If we assume that, all other things equal, Sue would rather see Joe in
jail than not, than if Sue is released she will always identify Joe to
the police. Hence, Joe shouldn't believe Sue's promise.

Now, suppose Sue makes an alternate promise: Sue commits a crime that
Joe will witness, then Joe will release Sue. We will assume that Sue
always commits the crime (and never instead runs away).

For this more complicated case, let's define some variables: let $k$ be
the cost to Joe for killing Sue; $c_J$ and $c_S$ be the costs to Joe and
Sue, respectively, for being identified to the police; and $b_J$ and
$b_S$ be the benefits to Joe and Sue, respectively, for identifying the
other to the police.

The game tree for this scenario is as follows:

\begin{center}
\begin{tikzpicture}[grow=down, sloped]
\node[bag] {Joe chooses}
    child {
        node[bag] {$(-k, - \infty)$}
            edge from parent
            node[above] {Kill}
    }
    child {
        node[bag] {Sue chooses}
        child {
                node[bag] {Joe chooses}
                child {
                        node[bag] {$(-c_J + b_J, -c_S + b_S)$}
                        edge from parent
                        node[above] {Identify}
                    }
                child {
                        node[bag] {$(-c_J, b_S)$}
                        edge from parent
                        node[above] {Don't identify}
                    }
                edge from parent
                node[above] {Identify}
            }
       child {
                node[bag] {Joe chooses}
                child {
                        node[bag] {Sue chooses}
                        child {
                                node[bag] {$(-c_J + b_J, -c_S + b_S)$}
                                edge from parent
                                node[above] {Identify}
                            }
                        child {
                                node[bag] {$(b_J, -c_S)$}
                                edge from parent
                                node[above] {Don't identify}
                            }
                        edge from parent
                        node[above] {Identify}
                    }
                child {
                        node[bag] {$(0, 0)$}
                        edge from parent
                        node[above] {Don't identify}
                    }
                edge from parent
                node[above] {Don't identify}
            }
            edge from parent
            node[above] {Release}
    };
\end{tikzpicture}
\end{center}

Using backwards induction we can infer that, so long as $-c_i + b_i >
-c_i$ (which might not be true if we had a case of Stockholm Syndrome or
something of the like), so long as one individual talks to the police,
so too will the other.

Hence, Sue will not identify Joe provided that $0 > -c_S + b_S$; that
is, so long as the benefit of identifying Joe is less than the cost of
being identified, Sue will not identify Joe.

\end{document}
