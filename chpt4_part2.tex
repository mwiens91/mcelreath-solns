% Set up the document
\documentclass{article}

% Page size
\usepackage[
    letterpaper,]{geometry}

% Lines between paragraphs
\setlength{\parskip}{\baselineskip}
\setlength{\parindent}{0pt}

% Math
\usepackage{amsfonts}
\usepackage{amsmath}

% Links
\usepackage{hyperref}

\begin{document}

\section*{6. Tolerance and $\boldsymbol{n}$-person reciprocity}

Recall that in the $n$-person iterated prisoner's dilemma ($n$-IPD),
$T_i$, $0 \leq i \leq n - 1$, is the strategy which always cooperates on
the first round, and only cooperates on subsequent rounds provided at
least $i$ other individuals cooperated on the previous round (else it
defects). In Chapter 4, we saw that $T_{n - 1}$ is an ESS against ALLD
for the $n$-IPD. We want to show that $T_{n - m}$ where $2 \leq m \leq
n$ is never an ESS for $n$-IPD against ALLD.

Fix $m$ such that $2 \leq m \leq n$. Assume we have a population where
$T_{n - m}$ is common. Then the relevant payoff to $T_{n - m}$ for the
$n$-IPD is

\begin{equation*}
    V(T_{n - m}|n - 1) = \frac{b - c}{1 - w}
\end{equation*}

while the relevant payoff for ALLD is

\begin{equation*}
    V(ALLD|n - 1) = \frac{b (n - 1)}{n} \cdot \frac{1}{1 - w}
\end{equation*}

Hence $T_{n - m}$ is an ESS against ALLD provided

\begin{align*}
    &V(T_{n - m}|n - 1) > V(ALLD|n - 1) \\
    &\Rightarrow \frac{b - c}{1 - w} > \frac{b (n - 1)}{n (1 - w)} \\
    &\Rightarrow n (b - c) > b (n - 1) \\
    &\Rightarrow - n c > - b \\
    &\Rightarrow c - \frac{b}{n} < 0
\end{align*}

Taking the limit as $n \to \infty$, we see that we end up with $c < 0$,
which is false! Hence $T_{n - m}$ is not an ESS against ALLD provided
$n$ is sufficiently large.

\section*{7. $\boldsymbol{n}$-person stag hunting}

Consider the $n$-person stag hunt game. In this game individuals can
play one of two strategies: hare hunting (H) or stag hunting (S).
Hunting hares is a solitary activity that yields payoff $h$ with success
rate $1$; hunting stags is a cooperative activity which has a payoff $s$
($s > h$) for each individual with success rate $\frac{x}{n}$ where $x$
is the number of stag hunters in the group.

Let $p$ be the frequency of S in the population. What we want to show
here is the conditions for which H and S are ESSs, and then calculate
any internal equilibria and their stability.

Let's first consider a population where H is common and S invades. The
stag-hunt game payoff for H is

\begin{equation*}
    V(H) = h
\end{equation*}

while the payoff for S is

\begin{equation*}
    V(S|0) = \frac{1}{n} s
\end{equation*}

Hence H is an ESS provided

\begin{align*}
    &V(H) > V(S|0) \\
    &\Rightarrow h > \frac{s}{n}
\end{align*}

Now let's consider a population where S is common and H invades. The
stag-hunt game payoff for H is the same as in the previous case, while
the payoff for S is now

\begin{equation*}
    V(S|n - 1) = s
\end{equation*}

Hence S is an ESS provided

\begin{align*}
    &V(S|n - 1) > V(H) \\
    &\Rightarrow s > h
\end{align*}

which is always true, by assumption.

Now let's consider internal equilibria. The payoff for H is going to
stay the same, so the fitness of H is given by

\begin{equation*}
    W(H) = w_0 + h
\end{equation*}

For the fitness of S, assume that the players in the game are randomly
selected from the population. This gives us

\begin{align*}
    W(S) &= w_0 + \sum_{i = 0}^{n - 1} \binom{n - 1}{i} p^i (1 - p)^{n - 1 - i} V(S|i) \\
         &= w_0 + \sum_{i = 0}^{n - 1} \binom{n - 1}{i} p^i (1 - p)^{n - 1 - i} \frac{i + 1}{n} s \\
         &= w_0 + \frac{s}{n} \sum_{i = 0}^{n - 1} \binom{n - 1}{i} p^i (1 - p)^{n - 1 - i} (i + 1) \\
         &= w_0 + \frac{s}{n} \left(\sum_{i = 0}^{n - 1} \binom{n - 1}{i} p^i (1 - p)^{n - 1 - i} i + \sum_{i = 0}^{n - 1} \binom{n - 1}{i} p^i (1 - p)^{n - 1 - i}\right) \\
         &= w_0 + \frac{s}{n} (p (n - 1) + 1)
\end{align*}

Let $\hat{p}$ denote the internal equilibrium frequency(s) of S. These
occur when

\begin{align*}
    &W(H) = W(S) \\
    &\Rightarrow w_0 + h = w_0 + \frac{s}{n} (\hat{p} (n - 1) + 1) \\
    &\Rightarrow h = \frac{s}{n} (\hat{p} (n - 1) + 1) \\
    &\Rightarrow \hat{p} = \frac{h n - s}{s (n - 1)}
\end{align*}

Note that this internal equilibrium exists provided that

\begin{equation*}
    h > \frac{s}{n}
\end{equation*}

But this is exactly our condition for H to be an ESS! Hence this
internal equilibrium is unstable if it exists (since both pure
equilibria will be stable).

\end{document}
