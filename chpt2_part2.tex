% Set up the document
\documentclass{article}

% Page size
\usepackage[
    letterpaper,]{geometry}

% Lines between paragraphs
\setlength{\parskip}{\baselineskip}
\setlength{\parindent}{0pt}

% Math
\usepackage{amsmath}

% Links
\usepackage{hyperref}

\begin{document}

\section*{4. Correlated versus uncorrelated symmetries}

Here we consider a mixed population of Bourgeois and Assessors. Recall
that in an interaction over a resource, Borgeois play Hawk when they
arrives first and play Dove when they arrives second; Assessors play
Hawk when it is stronger than its opponent and Dove when it is weaker.

Assume that the probability that a random Bourgeois is stronger than a
random Assesor is $\frac{1}{2}$. We start by calculating the payoff of
each possible interaction:

\begin{align*}
    V(B|B) &= \frac{v}{2} \\
    V(B|A) &= \frac{1}{2} V(H|A) + \frac{1}{2} V(D|A) \\
           &= \frac{1}{2} \cdot \Big[\frac{1}{2} v + \frac{1}{2}((1 - x) v - x c)\Big]
              + \frac{1}{2} \cdot \frac{v}{4} \\
           &= \frac{(5 - 2 x) v}{8} - \frac{x c}{4} \\
    V(A|B) &= \frac{1}{2} V(A|H) + \frac{1}{2} V(A|D) \\
           &= \frac{1}{2} \cdot \frac{x v - (1 - x) c}{2}
              + \frac{1}{2} \cdot \frac{3 v}{4} \\
           &= \frac{(x - 1) c}{4} + \frac{v (2 x + 3)}{8} \\
    V(A|A) &= \frac{v}{2}
\end{align*}

Now, consider what happens when rare Bourgeois invade a population of
Assessors:

In this case, the fitness of an Assesor is

\begin{equation*}
    W(A) \approx w_0 + \frac{v}{2}
\end{equation*}

and the fitness of a rare Bourgeois is

\begin{equation*}
    W(B) \approx w_0 + \frac{(5 - 2 x) v}{8} - \frac{x c}{4}
\end{equation*}

Assesor will be an ESS against Bourgeois provided

\begin{align*}
    &W(A) > W(B) \\
    &\Rightarrow w_0 + \frac{v}{2} > w_0 + \frac{(5 - 2 x) v}{8} - \frac{x c}{4} \\
    &\Rightarrow \frac{v}{2} > \frac{(5 - 2 x) v}{8} - \frac{x c}{4} \\
    &\Rightarrow x > \frac{v}{2(v + c)}
\end{align*}

Next, consider what happens when rare Assessors invade a population of Borgeois:

The fitness of a Borgeois is

\begin{equation*}
    W(A) \approx w_0 + \frac{v}{2}
\end{equation*}

and the fitness of a rare Assessor is

\begin{equation*}
    W(B) \approx w_0 + \frac{(x - 1) c}{4} + \frac{v (2 x + 3)}{8} \\
\end{equation*}

Bourgeois will be an ESS against Assessor provided

\begin{align*}
    &W(B) > W(A) \\
    &\Rightarrow w_0 + \frac{v}{2} > w_0 + \frac{(x - 1) c}{4} + \frac{v (2 x + 3)}{8} \\
    &\Rightarrow \frac{v}{2} > \frac{(x - 1) c}{4} + \frac{v (2 x + 3)}{8} \\
    &\Rightarrow x < \frac{v + 2 c}{2(v + c)}
\end{align*}

Suppose that

\begin{equation*}
    \frac{v}{2(v + c)} < x < \frac{v + 2 c}{2(v + c)}
\end{equation*}

Hence, if a single mixed equilibrium exists, it will be unstable (since
both pure equilibria are stable). Let $p$ be the frequency of the
population that are Assessors.

Then, in a mixed population,

\begin{align*}
    W(B) &= w_0 + p V(B|A) + (1 - p) V(B|B) \\
         &= w_0 + p \Big[\frac{(5 - 2 x) v}{8} - \frac{x c}{4}\Big]
            + (1 - p) \frac{v}{2} \\
    W(A) &= w_0 + p V(A|A) + (1 - p) V(A|B) \\
         &= w_0 + p \frac{v}{2}
            + (1 - p) \Big[\frac{(x - 1) c}{4} + \frac{v (2 x + 3)}{8}\Big]
\end{align*}

Let $\hat{p}$ be the equilibrium frequency of Assessors in the mixed
population. Then we have that

\begin{align*}
    &W(B) = W(A) \\
    &\Rightarrow
        w_0 + \hat{p} \Big[\frac{(5 - 2 x) v}{8} - \frac{x c}{4}\Big]
        + (1 - \hat{p}) \frac{v}{2}
        =
        w_0 + \hat{p} \frac{v}{2}
        + (1 - \hat{p}) \Big[\frac{(x - 1) c}{4} + \frac{v (2 x + 3)}{8}\Big] \\
    &\Rightarrow
        \hat{p} \Big[\frac{(5 - 2 x) v}{8} - \frac{x c}{4}\Big]
        + (1 - \hat{p}) \frac{v}{2}
        =
        \hat{p} \frac{v}{2}
        + (1 - \hat{p}) \Big[\frac{(x - 1) c}{4} + \frac{v (2 x + 3)}{8}\Big] \\
    &\Rightarrow \hat{p} = \frac{v (1 - 2 x) + 2 c (1 - x)}{2 c}
\end{align*}

We can read off from the expression of $\hat{p}$ above that

\begin{equation*}
    \hat{p}\bigr|_{x=0.5} = \frac{1}{2}
\end{equation*}

and also that when $x > 0.5$, $\hat{p} < \frac{1}{2}$ (provided that the
equation for $\hat{p}$ satisfies probability constraints---i.e., $0 \leq
\hat{p} \leq 1$). This can be heuristically explained by considering
that when $x$ is small, Assessor's assessment of potential combats are
accurate; so they play Hawk for combats they can win and Dove otherwise.
However when $x$ is large, Assessor's assessments of potential combats
aren't very accurate; they end up shying away (playing Dove) in combats
that they could have otherwise (probably) won, and hence lose out on
all/a greater share of the resource at stake.

\end{document}
